\documentclass{article}
	
\usepackage[margin=1in]{geometry}		% For setting margins
\usepackage{amsmath}				% For Math
\usepackage[]{amssymb}
\usepackage{amsmath}
\usepackage{gensymb}
\usepackage{fancyhdr}				% For fancy header/footer
\usepackage{graphicx}				% For including figure/image
\usepackage{cancel}					% To use the slash to cancel out stuff in work
\usepackage{wasysym}                % For cent symbol

%%%%%%%%%%%%%%%%%%%%%%
% Set up fancy header/footer
\pagestyle{fancy}
\fancyhead[RO,R]{{\large\textbf{Andry Paez}}}
\fancyhead[LO,L]{\large{\textbf{Ch 19 - Heat and the First Law of Thermodynamics Part 1}}}
% \fancyhead[RO,R]{\today}
\fancyfoot[LO,L]{}
\fancyfoot[CO,C]{\thepage}
\fancyfoot[RO,R]{}
\renewcommand{\headrulewidth}{0.4pt}
\renewcommand{\footrulewidth}{0.4pt}
%%%%%%%%%%%%%%%%%%%%%%

\begin{document}
\begin{center}
    \section*{\textbf{\underline {Conceptual Questions}}}
\end{center}

\subsection*{2. When a hot object warms a cooler object, does temperature flow between them? Are the temperature changes of the two objects equal? Explain.} \\

\subsection*{5. The specific heat of water is quite large. Explain why this fact maker water particularly good for heating systems (that is, hot-water radiators)} \\

\subsection*{7. Explain why burns caused by steam at 100$\degree$C on the skin are often more sever than burns caused by water at 100$\degree$C.} \\

\subsection*{10. Very high in the Earth's atmosphere the temperature can be 700$\degree$C. Yet an animal there would freeze to death rather than roast. Explain.} \\
\hrule \vspace{2em}
\begin{center}
    \section*{\textbf{\underline {Problems}}}
\end{center}
\large{\textbf{\textit{19-1 Heat as Energy Transfer}}}

\subsection*{1. To what temperature will 8700 J of heat raise 3.0 kg of water that is initially at 10.0$\degree$C?}

\subsection*{3. An average active person consumes about 2500 Cal a day. (\textit a) What is this in joules? (\textit b) What is this in kilowatt-hours? (\textit c) If your power company charges about 10$\cent$ per kilowatt-hour, how much would your energy cost per day if you bought it from the power company? Could you feed yourself on this much money per day?}

\subsection*{5. How Many joules and kilocalories are generated when the brakes are used to bring a 1200-kg car to rest from a speed of 95 km/h?}
\vspace{1em}
\large{\textbf{\textit{19-3 and 19-4 Specific Heat; Calorimetry}}}

\subsection*{7. An automobile cooling system holds 18 L of water. How much heat does it absorb if its temperature rises from 15$\degree$C to 95$\degree$C?}

\subsection*{9. (\textit a) How much energy is required to bring a 1.0-L pot of water at 20$\degree$C to 100$\degree$C? (\textit b) For how long could this amount of energy run a 100-W lightbulb?}

\subsection*{11. How long does it take a 750-W coffeepot to bring to a boil 0.75 L of water initially at 8.0$\degree$C? Assume that the part of the pot which is heated with the water is made of 280g of aluminum, and that no water boils away}

\subsection*{13. A 31.5-g glass thermometer reads 23.6$\degree$C before it is placed in 135 mL of water. When the water and thermometer come to equilibrium, the thermometer reads 39.2$\degree$C. What was the original temperature of the water? [\textit{Hint}: Ignore the mass of fluid inside the glass thermometer.]}
\]
\subsection*{15. When a 290-g piece of iron at 180$\degree$C is placed in a 95-g aluminum calorimeter cup containing 250g of glycerin at 10$\degree$C, the final temperature is observed to be 38$\degree$C. Estimate the specific heat of glycerin.}

\subsection*{17. The 1.20-kg head of a hammer has a speed of 7.5 m/s just before it strikes a nail (Fig.19-29) and is brought to rest. Estimate the temperature rise of a 14-g iron nail generated by 10 such hammer blows done in quick succession. Assume the nail absorbs all the energy.}
\vspace{1em}
\large{\textbf{\textit{19-5 Latent Heat}}}

\subsection*{18. How much heat is needed to melt 26.50 kg of silver that is initially at 25$\degree$C?}

\subsection*{20. A 35-g ice cube at its melting pint is dropped into an insulated container of liquid nitrogen. How much nitrogen evaporates if it is at its boiling point of 77K and has a latent heat of vaporization of 200 kJ/kg? Assume for simplicity that the specific heat of ice is constant and is equal to its value near its melting point.}

\subsection*{22. An iron boiler of mass 180 kg contains 730kg of water at 18$\degree$C. A heater supplies energy at the rate of 52,000 kJ/h. How long does it take for the water (\textit a) to reach the boiling point, and (\textit b) to all have changed to steam?}

\end{document}
